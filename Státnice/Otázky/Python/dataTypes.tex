\documentclass{article}
\usepackage{listings} % For code listings
\usepackage{xcolor} % For defining colors

% Define colors for code listing
\definecolor{codegreen}{rgb}{0,0.6,0}
\definecolor{codegray}{rgb}{0.5,0.5,0.5}
\definecolor{codepurple}{rgb}{0.58,0,0.82}
\definecolor{backcolour}{rgb}{0.95,0.95,0.92}

% Define code listing style
\lstdefinestyle{mystyle}{
	backgroundcolor=\color{backcolour},
	commentstyle=\color{codegreen},
	keywordstyle=\color{magenta},
	numberstyle=\tiny\color{codegray},
	stringstyle=\color{codepurple},
	basicstyle=\ttfamily\footnotesize,
	breakatwhitespace=false,
	breaklines=true,
	captionpos=b,
	keepspaces=true,
	numbers=left,
	numbersep=5pt,
	showspaces=false,
	showstringspaces=false,
	showtabs=false,
	tabsize=2
}

\lstset{style=mystyle}

\begin{document}
	
	\section{Introduction}
	In computer programming, data types specify the type of data that can be stored inside a variable. For example,
	\begin{lstlisting}[language=Python]
		num = 24
	\end{lstlisting}
	Here, 24 (an integer) is assigned to the \texttt{num} variable. So the data type of \texttt{num} is of the \texttt{int} class.
	
	\section{Python Data Types}
	\begin{table}[htbp]
		\centering
		\begin{tabular}{|l|l|p{6cm}|}
			\hline
			\textbf{Data Types} & \textbf{Classes} & \textbf{Description} \\ \hline
			Numeric & int, float, complex & Holds numeric values \\ \hline
			String & str & Holds a sequence of characters \\ \hline
			Sequence & list, tuple, range & Holds a collection of items \\ \hline
			Mapping & dict & Holds data in key-value pair form \\ \hline
			Boolean & bool & Holds either True or False \\ \hline
			Set & set, frozenset & Holds a collection of unique items \\ \hline
		\end{tabular}
		\caption{Python Data Types}
		\label{tab:datatypes}
	\end{table}
	
	Since everything is an object in Python programming, data types are actually classes and variables are instances (objects) of these classes.
	
	\section{Python Numeric Data Type}
	In Python, the numeric data type is used to hold numeric values.
	\begin{lstlisting}[language=Python]
		num1 = 5
		print(num1, 'is of type', type(num1))
		
		num2 = 2.0
		print(num2, 'is of type', type(num2))
		
		num3 = 1+2j
		print(num3, 'is of type', type(num3))
	\end{lstlisting}
	Output:
	\begin{lstlisting}[language=Python]
		5 is of type <class 'int'>
		2.0 is of type <class 'float'>
		(1+2j) is of type <class 'complex'>
	\end{lstlisting}
	In the above example, we have created three variables named \texttt{num1}, \texttt{num2}, and \texttt{num3} with values 5, 5.0, and 1+2j respectively.
	
	\section{Python List Data Type}
	List is an ordered collection of similar or different types of items separated by commas and enclosed within brackets \texttt{[]}. For example,
	\begin{lstlisting}[language=Python]
		languages = ["Swift", "Java", "Python"]
	\end{lstlisting}
	...

\section{Python List Data Type (Continued)}
List is an ordered collection of similar or different types of items separated by commas and enclosed within brackets \texttt{[]}. For example,
\begin{lstlisting}[language=Python]
	languages = ["Swift", "Java", "Python"]
\end{lstlisting}

\subsection{Access List Items}
To access items from a list, we use the index number (0, 1, 2, ...). For example,
\begin{lstlisting}[language=Python]
	languages = ["Swift", "Java", "Python"]
	
	% access element at index 0
	print(languages[0])   % Swift
	
	% access element at index 2
	print(languages[2])   % Python
\end{lstlisting}
In the above example, we have used the index values to access items from the \texttt{languages} list.

\section{Python Tuple Data Type}
Tuple is an ordered sequence of items same as a list. The only difference is that tuples are immutable. Tuples once created cannot be modified.

In Python, we use the parentheses \texttt{()} to store items of a tuple. For example,
\begin{lstlisting}[language=Python]
	product = ('Xbox', 499.99)
\end{lstlisting}

\subsection{Access Tuple Items}
Similar to lists, we use the index number to access tuple items in Python. For example,
\begin{lstlisting}[language=Python]
	% create a tuple 
	product = ('Microsoft', 'Xbox', 499.99)
	
	% access element at index 0
	print(product[0])   % Microsoft
	
	% access element at index 1
	print(product[1])   % Xbox
\end{lstlisting}

\section{Python String Data Type}
String is a sequence of characters represented by either single or double quotes. For example,
\begin{lstlisting}[language=Python]
	name = 'Python'
	print(name)  
	
	message = 'Python for beginners'
	print(message)
\end{lstlisting}

\section{Python Set Data Type}
Set is an unordered collection of unique items. Set is defined by values separated by commas inside braces \texttt{\{\}}. For example,
\begin{lstlisting}[language=Python]
	% create a set named student_id
	student_id = {112, 114, 116, 118, 115}
	
	% display student_id elements
	print(student_id)
	
	% display type of student_id
	print(type(student_id))
\end{lstlisting}

\section{Python Dictionary Data Type}
Python dictionary is an ordered collection of items. It stores elements in key/value pairs.

\subsection{Access Dictionary Values Using Keys}
We use keys to retrieve the respective value. But not the other way around. For example,
\begin{lstlisting}[language=Python]
	% create a dictionary named capital_city
	capital_city = {'Nepal': 'Kathmandu', 'Italy': 'Rome', 'England': 'London'}
	
	print(capital_city['Nepal'])  % prints Kathmandu
	
	% Throws error message
	print(capital_city['Kathmandu'])
\end{lstlisting}

\section{Python Type Conversion}
In programming, type conversion is the process of converting data of one type to another. There are two types of type conversion in Python.

\subsection{Implicit Conversion}
In certain situations, Python automatically converts one data type to another. This is known as implicit type conversion.

\subsubsection{Example 1: Converting integer to float}
Let's see an example where Python promotes the conversion of the lower data type (integer) to the higher data type (float) to avoid data loss.
\begin{lstlisting}[language=Python]
	integer_number = 123
	float_number = 1.23
	
	new_number = integer_number + float_number
	
	% display new value and resulting data type
	print("Value:",new_number)
	print("Data Type:",type(new_number))
\end{lstlisting}
Output:
\begin{lstlisting}[language=Python]
	Value: 124.23
	Data Type: <class 'float'>
\end{lstlisting}
In the above example, we have created two variables: \texttt{integer\_number} and \texttt{float\_number} of \texttt{int} and \texttt{float} type respectively.

Then we added these two variables and stored the result in \texttt{new\_number}.

As we can see, \texttt{new\_number} has the value 124.23 and is of the \texttt{float} data type.

\subsubsection{Note}
We get \texttt{TypeError}, if we try to add \texttt{str} and \texttt{int}. For example, '12' + 23. Python is not able to use Implicit Conversion in such conditions.

\subsection{Explicit Conversion}
In Explicit Type Conversion, users convert the data type of an object to the required data type.

We use the built-in functions like \texttt{int()}, \texttt{float()}, \texttt{str()}, etc to perform explicit type conversion.

This type of conversion is also called typecasting because the user casts (changes) the data type of the objects.

\subsubsection{Example 2: Addition of string and integer Using Explicit Conversion}
\begin{lstlisting}[language=Python]
	num_string = '12'
	num_integer = 23
	
	print("Data type of num_string before Type Casting:",type(num_string))
	
	% explicit type conversion
	num_string = int(num_string)
	
	print("Data type of num_string after Type Casting:",type(num_string))
	
	num_sum = num_integer + num_string
	
	print("Sum:",num_sum)
	print("Data type of num_sum:",type(num_sum))
\end{lstlisting}
Output:
\begin{lstlisting}[language=Python]
	Data type of num_string before Type Casting: <class 'str'>
	Data type of num_string after Type Casting: <class 'int'>
	Sum: 35
	Data type of num_sum: <class 'int'>
\end{lstlisting}
In the above example, we have created two variables: \texttt{num\_string} and \texttt{num\_integer} with \texttt{str} and \texttt{int} type values respectively. Notice the code,

\texttt{num\_string = int(num\_string)}

Here, we have used \texttt{int()} to perform explicit type conversion of \texttt{num\_string} to integer type.

After converting \texttt{num\_string} to an integer value, Python is able to add these two variables.

Finally, we got the \texttt{num\_sum} value i.e 35 and data type to be \texttt{int}.

\subsection{Key Points to Remember}
\begin{itemize}
	\item Type Conversion is the conversion of an object from one data type to another.
	\item Implicit Type Conversion is automatically performed by the Python interpreter.
	\item Python avoids the loss of data in Implicit Type Conversion.
	\item Explicit Type Conversion is also called Type Casting, the data types of objects are converted using predefined functions by the user.
	\item In Type Casting, loss of data may occur as we enforce the object to a specific data type.
\end{itemize}

\end{document}