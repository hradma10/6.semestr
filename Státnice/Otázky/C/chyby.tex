\documentclass{article}
\usepackage[czech]{babel} % pro češtinu
\usepackage{listings}

\begin{document}
	
	\section*{Typy chyb a jejich hledání v jazyce C}
	
	1. \textbf{Syntaxové chyby:}
	
	\textbf{Příklad:}
	\begin{verbatim}
		int main() {
			printf("Hello, world!\n");
		}
	\end{verbatim}
	Tento kód má syntaxovou chybu ve formě chybějícího zahrnutí hlavičkového souboru <stdio.h>. Kompilátor vygeneruje chybovou zprávu, která upozorní na tento problém.
	
	\textbf{Hledání:} Syntaxové chyby jsou obvykle detekovány kompilátorem, který vydá chybové zprávy při kompilaci kódu. Stačí procházet chybové zprávy, abyste našli a opravili syntaxové chyby.
	
	2. \textbf{Logické chyby:}
	
	\textbf{Příklad:}
	\begin{verbatim}
		int sum(int a, int b) {
			return a - b; // Chybně odečítáme místo sčítáme
		}
	\end{verbatim}
	Tento kód má logickou chybu, kde funkce \texttt{sum} odečítá \texttt{b} od \texttt{a} místo aby je sčítala.
	
	\textbf{Hledání:} Logické chyby jsou obtížnější najít, protože program kompiluje a spouští se bez chybějících chybových zpráv. K hledání logických chyb lze použít ladící nástroje, jako je GDB, a testování vstupů programu.
	
	3. \textbf{Chyby za běhu programu:}
	
	\textbf{Příklad:}
	\begin{verbatim}
		int main() {
			int a = 10, b = 0;
			int result = a / b; // Dělení nulou
			printf("Result: %d\n", result);
			return 0;
		}
	\end{verbatim}
	Tento kód má chybu za běhu programu, protože dělíme \texttt{a} nulou, což způsobí dělení nulou.
	
	\textbf{Hledání:} Chyby za běhu programu mohou být detekovány pomocí ladících nástrojů, jako je GDB. Ladění umožňuje programátorovi sledovat chování programu v průběhu jeho běhu a identifikovat problémové části kódu.
	
	4. \textbf{Chyby přístupu do paměti:}
	
	\textbf{Příklad:}
	\begin{verbatim}
		int main() {
			int arr[5] = {1, 2, 3, 4, 5};
			printf("%d\n", arr[10]); // Přístup mimo rozsah pole
			return 0;
		}
	\end{verbatim}
	Tento kód má chybu přístupu do paměti, protože se pokoušíme přistoupit k prvkům pole mimo jeho rozsah.
	
	\textbf{Hledání:} Chyby přístupu do paměti mohou být obtížné najít, ale lze je identifikovat pomocí nástrojů pro sledování paměti, jako je Valgrind. Tyto nástroje sledují přístupy do paměti a upozorňují na neplatné operace.
	
	5. \textbf{Chyby ve správě paměti:}
	
	\textbf{Příklad:}
	\begin{verbatim}
		int main() {
			int *ptr = malloc(10 * sizeof(int));
			ptr = NULL; // Ztráta odkazu na alokovanou paměť
			free(ptr); // Pokus o uvolnění nulového ukazatele
			return 0;
		}
	\end{verbatim}
	Tento kód má chyby ve správě paměti: ztrátu odkazu na alokovanou paměť a pokus o uvolnění nulového ukazatele.
	
	\textbf{Hledání:} Chyby ve správě paměti mohou být identifikovány pomocí nástrojů pro sledování paměti, jako je Valgrind nebo AddressSanitizer. Tyto nástroje sledují alokaci a uvolňování paměti a upozorňují na potenciální problémy.
	
\section*{Časté chyby}

Pro snažší identifikaci programovacích defektů je praktické mít seznam s jednoznačnými identifikátory. Podobně jako pro bezpečnostní problémy existuje výčet Common Vulnerabilities and Exposures (CVE)\footnote{http://cve.mitre.org/}, vznikl i seznam Common Weakness Enumeration (CWE)\footnote{http://cwe.mitre.org/} spravovaný Mitre Corporation. Tento seznam obsahuje jak možné problémy v implementaci počítačových programů, tak i jejich návrhu a architektuře. Každému defektu je věnována stránka s popisem problému a určením, ve které části vzniku softwaru se chyba objevuje. Dále následuje výčet možných důsledků, příklady chybných programů s vysvětlením, co a proč je v nich špatně, a návrhy, jak tomuto problému předcházet, případně jak ho odhalit. Všechny stránky na webu CWE jsou řazeny do stromové hierarchie, takže je možné se snadno z popisu jednoho defektu dostat k podobným chybám, případně obecnému popisu skupiny chyb.

Samotný web by mohl v budoucnu sloužit nejen jako databáze znalostí o programovacích chybách. Jeho potenciál je větší. Pro usnadnění práce s nástroji pro statickou analýzu by bylo velmi vhodné standardizovat sadu anotací, pomocí kterých by bylo možné vyznačit ve zdrojovém kódu problematické úseky a tím dát najevo, že daný řádek kódu nebo funkce opravdu má vypadat tak, jak vypadá, a nejde o chybu.

V následujících odstavcích jsou představeny některé z možných problémů.

\subsection*{Použití neinicializované proměnné – CWE-457}

V jazyku C i C++ se nově deklarovaným proměnným automaticky nepřiřazuje žádná výchozí hodnota. Obsahují tedy to, co bylo v paměti na jejich adrese uloženo dřív. Tato data jsou jen velmi vzácně k něčemu užitečná a je tedy praktické co nejdříve proměnným přiřadit nějakou jasně definovanou hodnotu. Neinicializované proměnné mohou vést k pádu programu. Pokud by se potenciálnímu útočníkovi podařilo zajistit, aby na adrese neinicializované proměnné byla jím připravená data, mohl by ovlivnit běh programu.

\textbf{Příklad:}

	
\begin{verbatim}
	int x; // neinicializovaná proměnná
	printf("Hodnota x: \%d\\n", x); // může obsahovat nedefinovanou hodnotu
\end{verbatim}
	\subsection*{Dereference ukazatele NULL – CWE-476}
	
	Dereference ukazatele NULL obvykle vede k pádu programu se signálem SIGSEGV. K této chybě může dojít jak souběhem vláken se špatnou synchronizací, tak i prostým opomenutím programátora. Obvykle se chyba projevuje v částech kódu, které se nepoužívají příliš často a tedy mohou snadněji projít testováním neodhaleny.
	
\begin{verbatim}
	int *ptr = NULL;
	*ptr = 5; // Dereference NULL pointer
\end{verbatim}
	
\subsection*{Únik paměti – CWE-401}

Pokud program korektně nesleduje a neuvolňuje alokovanou paměť, může docházet k postupnému zvyšování spotřeby paměti. To vede k nižší spolehlivosti programu.

\begin{verbatim}
void function() {
	int *ptr = malloc(sizeof(int));
	// paměť není uvolněna, dochází k úniku paměti
}
\end{verbatim}
\subsection*{Únik deskriptorů – CWE-775}

Podobný případ neuvolňování paměti je i situace, kdy program otevře např. soubor, ale po skončení práce s ním ho už nezavře. Přestože po skončení programu operační systém všechny otevřené soubory zavře, v případě větších a déle běžících systémů může snadno dojít k vyčerpání všech dostupných deskriptorů a následnému selhání při otevírání dalšího souboru.

\textbf{Příklad:}

\begin{verbatim}
FILE *fp = fopen("file.txt", "r");
// neprovádí se fclose(fp), což vede k úniku deskriptoru
\end{verbatim}

\section*{Použití paměti po uvolnění – CWE-416}

Použití paměti, která byla uvolněna, může vést k pádu programu, použití neočekávané hodnoty nebo i spuštění libovolného kódu. Při přístupu k uvolněné paměti může dojít k poškození dat, pokud už byla stejná paměť alokována jinde.

\textbf{Příklad:}
\begin{verbatim}
	int *ptr = malloc(sizeof(int));
	free(ptr);
	*ptr = 5; // použití uvolněné paměti
\end{verbatim}

\section*{Mrtvý kód – CWE-561}

Mrtvý kód (dead code) je takový kód, který se nikdy nevykoná. Přestože se technicky nejedná o chybu a na běh programu nemá vliv, může takový kód komplikovat údržbu programu a tím brzdit opravu jiných, už závažnějších chyb. Důvodem, proč se některá část kódu nikdy neprovede, může být například testování podmínky, která je vždy pravdivá (resp. nepravdivá).

\textbf{Příklad:}
\begin{verbatim}
	void function(int x) {
		if (x > 0) {
			// mrtvý kód, protože x nikdy nebude záporné
		} else {
			// tento blok kódu se vždy vykoná
		}
	}
\end{verbatim}

\section*{Race condition – CWE-362}

Race condition je situace, kdy výsledek nějaké operace je závislý na pořadí a načasování jednotlivých operací vícevláknových programů. Souběh obvykle bývá způsoben chybným přístupem ke sdílenému prostředku, např. modifikací datové struktury z více vláken zároveň.

\textbf{Příklad:}
\begin{verbatim}
	int counter = 0;
	
	void increment_counter() {
		counter++;
	}
	
	// vlákno 1
	increment_counter();
	
	// vlákno 2
	increment_counter();
	
	// Výsledek není deterministický kvůli nedostatečné synchronizaci
\end{verbatim}

\section*{Chyby zámků – CWE-667}

Při používání sdíleného zdroje více vlákny programu je obvykle třeba zajistit, aby zdroj mohl být v jednom okamžiku používán pouze jedním vláknem. K tomu je mimo jiné možné používat zámky, mutexy a semafory. Existuje mnoho různých konkrétních chyb při práci se zámky, jako je zapomenutí nutnosti získat zámek, neošetření selhání synchronizační funkce, deadlock atd.

\textbf{Příklad:}
\begin{verbatim}
	#include <pthread.h>
	
	pthread_mutex_t lock = PTHREAD_MUTEX_INITIALIZER;
	
	void *thread_func(void *arg) {
		pthread_mutex_lock(&lock);
		// zde může být kód, který manipuluje se sdílenými daty
		pthread_mutex_unlock(&lock);
		return NULL;
	}
\end{verbatim}

\section*{Přetečení bufferu – CWE-120}

K přetečení bufferu dochází tehdy, když program zkopíruje data ze vstupního bufferu do výstupního, aniž by zkontroloval, že velikost vstupního bufferu není větší než velikost výstupního bufferu.

\textbf{Příklad:}
\begin{verbatim}
	#include <string.h>
	
	void copy_string(char *src) {
		char dst[10];
		strcpy(dst, src); // Přetečení bufferu muze nastat, pokud src obsahuje více nez 10 znaku
	}
\end{verbatim}
	
\end{document}
