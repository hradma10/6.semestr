\documentclass{article}
\usepackage[utf8]{inputenc}
\usepackage{enumitem}


\DeclareUnicodeCharacter{2713}{\checkmark}
\usepackage{newunicodechar}

\usepackage{tikz}
\def\checkmark{\tikz\fill[scale=0.4](0,.35) -- (.25,0) -- (1,.7) -- (.25,.15) -- cycle;} 

\begin{document}
	
	
	
	\section*{Teoretické Základy informačních technologií}
	\subsection*{DISK1}
	\begin{enumerate}[label=\arabic*.]
		\item Výroková logika, formule, pravdivost, vyplývání \checkmark \checkmark
		\item Booleovské funkce, funkčně úplné systémy \checkmark \checkmark
		\item Úplné konjunktivní a disjunktivní normální formy \checkmark \checkmark
		\item Množiny, množinové operace, potenční množina, kartézský součin, číselné a nespočetné množiny \checkmark \checkmark
		\item Relace, binární relace a jejich reprezentace, operace s relacemi \checkmark \checkmark
		\item Funkce (zobrazení) a jejich vlastnosti \checkmark \checkmark
		\item Binární relace na množině a jejich vlastnosti \checkmark \checkmark
		\item Ekvivalence a rozklady \checkmark \checkmark
		\item Uspořádání, Hasseovy diagramy \checkmark \checkmark
		\item Permutace, variace, kombinace \checkmark \checkmark
		\item Pravděpodobnost, Laplaceova definice, pravděpodobností prostor, náhodná veličina, střední hodnota \checkmark \checkmark
		\item Indukce a rekurze, matematická indukce a její varianty \checkmark \checkmark
		\item Orientované a neorientované grafy, základní pojmy \checkmark \checkmark
		\item Hledání nejkratší cesty, Dijkstrův algoritmus \checkmark \checkmark
		\item Minimální kostra grafu, Kruskalův algoritmus \checkmark \checkmark
		\item Stromy, kořenové stromy, vztahy mezi výškou, počtem vrcholů a počtem listů \checkmark \checkmark
	\end{enumerate}
	
	Postup DISK1: 16/16 100\% \checkmark \checkmark
	
	
	
	\subsection*{IMAT1}
	\begin{enumerate}[label=\arabic*.]
		\item Matice, operace s maticemi, hodnost, determinant \checkmark \checkmark
		\item Vektorové prostory podprostory, báze a dimenze, matice přechodu \checkmark \checkmark
		\item Eukleidovské vektorové prostory, ortogonální a ortonormální báze, Schwarzova nerovnost, Schmidtova ortogonalizační metoda \checkmark \checkmark
		\item Soustavy lineárních rovnic, Frobeniova věta, Gaussova eliminační metoda, Cramerovo pravidlo \checkmark \checkmark
		\item Lineární zobrazení a transformace a jejich matice \checkmark \checkmark
	\end{enumerate}
	
	Postup IMAT1: 5/5 100\% \checkmark \checkmark

	\subsection*{IMAT2}
	\begin{enumerate}[label=\arabic*.]
		\item Funkce jedné reálné proměnné, základní vlastnosti \checkmark \checkmark
		\item Posloupnosti a jejich limity, limes superior, limes inferior \checkmark \checkmark
		\item Limita funkce včetně nevlastních, jednostranné limity \checkmark \checkmark
		\item Spojitost funkce: spojitost v bodě, spojitost na intervalu \checkmark \checkmark
		\item Vlastnosti spojitých funkcí, spojitost složené a inverzní funkce \checkmark \checkmark
		\item Derivace funkce a její geometrický význam \checkmark \checkmark
		\item Pravidla pro derivování funkce, derivace složené funkce, derivace inverzní funkce, derivace elementárních funkcí \checkmark \checkmark
		\item Průběh funkce: základní věty diferenciálního počtu, extrémy určitý funkce, konvexní a konkávní křivky, asymptoty \checkmark \checkmark
		\item Neurčitý Integrál a metody jeho výpočtu \checkmark \checkmark
		\item Riemannův určitý integrál: definice, základní věta integrálního počtu, metody výpočtu \checkmark \checkmark
		\item Geometrická interpretace určitého integrálu \checkmark \checkmark
	\end{enumerate}
	
	Postup IMAT2: 11/11 100\% \checkmark \checkmark

	\subsection*{ALGO}
	\begin{enumerate}[label=\arabic*.]
		\item Algoritmus, problém, časová složitost algoritmu v nejhorším a průměrném případě \checkmark \checkmark
		\item O-notace a růst funkcí, definice, vlastnosti, příklady \checkmark \checkmark
		\item Lineární datové struktury: Seznam, Zásobník, Fronta \checkmark \checkmark
		\item Problém třídění, rozdělení třídících algoritmů, dolní mez složitosti \checkmark \checkmark
		\item Základní algoritmy třídění 1: třídění porovnáváním, insert sort \checkmark \checkmark
		\item Základní algoritmy třídění 2: select sort, bubble sort \checkmark \checkmark
		\item Základní algoritmy třídění 3: quick sort a složitost \checkmark \checkmark
		\item Základní algoritmy třídění 4: merge sort a složitost \checkmark \checkmark
		\item Základní algoritmy třídění 5: heap sort a složitost \checkmark \checkmark
		\item Základní algoritmy třídění 6: counting sort, radix sort, bucket sort \checkmark \checkmark
		\item Pořádková statistika \checkmark \checkmark
	\end{enumerate}
	
	Postup ALGO: 11/11 100\% \checkmark \checkmark

	\subsection*{ZADS}
	\begin{enumerate}[label=\arabic*.]
		\item Vyhledávání v lineárních datových strukturách \checkmark \checkmark
		\item Binární vyhledávací stromy \checkmark \checkmark
		\item Binární vyhledávací stromy - operace a jejich složitost \checkmark \checkmark
		\item AVL stromy, operace, složitost \checkmark \checkmark
		\item B stromy, operace a jejich složitost \checkmark \checkmark
		\item Hashovací tabulky, metody řešení kolizí \checkmark \checkmark
		\item Základní grafové algoritmy: průchod do šířky \checkmark \checkmark
		\item Základní grafové algoritmy: průchod do hloubky \checkmark \checkmark
		\item Základní grafové algoritmy: topologické uspořádání \checkmark \checkmark
	\end{enumerate}
	
	Postup ZADS: 9/9 100\% \checkmark \checkmark
	\newline
	\newline
	Postup Teoretické Základy informačních technologií: 52/52 100\% \checkmark \checkmark
	\newline
	\newline
	1.pokus Teoretické Základy informačních technologií \(\times\) 
	
	\section*{Informační technologie}
	
	\subsection*{OS1}
	
	\begin{enumerate}[label=\arabic*.]
		\item Operační systém, x86 architektura \checkmark \checkmark
		\item x86, přístup k paměti \checkmark \checkmark
		\item Cdecl volací konvence \checkmark \checkmark
		\item Podmíněné skoky, přerušení, DMA \checkmark \checkmark
		\item Rozšíření x86 \checkmark \checkmark
		\item Instrukční sady dalších procesorů \checkmark \checkmark
		\item Vykonávání programu a proces překladu \checkmark \checkmark
		\item Architektura operačních systémů \checkmark \checkmark
		\item Správa procesoru: procesy a vlákna \checkmark \checkmark
		\item Správa procesoru: plánování běhu procesů a vláken \checkmark \checkmark
		\item Komunikace a synchronizace procesů a vláken \checkmark \checkmark
		\item Problém uváznutí, jeho detekce a metody předcházení \checkmark \checkmark
	\end{enumerate}
	
	Postup OS1: 12/12 100\% \checkmark \checkmark
	
	\subsection*{OS2}
	
	\begin{enumerate}[label=\arabic*.]
		\item Operační paměť, stránkování \checkmark \checkmark
		\item Virtuální paměť \checkmark \checkmark
		\item Implementace stránkování na i386 \checkmark \checkmark
		\item Implementace stránkování na AMD64, ostatní \checkmark \checkmark
		\item Správa paměti, manuálně \checkmark \checkmark
		\item Správa paměti, GC \checkmark \checkmark
		\item IPC \checkmark \checkmark
		\item Práce se I/O zařízení, ovladače \checkmark \checkmark
		\item Souborové systémy - obecně \checkmark \checkmark
		\item FAT, UFS, NTFS \checkmark \checkmark
		\item LVM, zbývající souborové systémy \checkmark \checkmark
		\item Bezpečnost \checkmark \checkmark
	\end{enumerate} 
	
	Postup OS2: 12/12 100\% \checkmark \checkmark
	
	\subsection*{DATAB}
	
	\begin{enumerate}[label=\arabic*.]
		\item Tabulky v SQL a jejich vztah k relacím \checkmark \checkmark
		\item Výraz SELECT v SQL - základy \checkmark \checkmark
		\item Výraz SELECT v SQL - group by , order by, limit, offset \checkmark \checkmark
		\item Relační operace: sjednocení, průnik, rozdíl, restrikce, projekce, přirozené spojení, přejmenování atributů \checkmark \checkmark
		\item Integrita dat: primární a cizí klíč \checkmark \checkmark
		\item Dokumentový model databáze: dokumenty, kolekce, atomické hodnoty, pole \checkmark \checkmark
		\item Základy práce v MongoDB: identifikátor dokumentu, operátory v dotazech, implicitní operátory \checkmark \checkmark
		\item Základy práce v MongoDB: dotazy na vnořené dokumenty \checkmark \checkmark
		\item Elasticsearch: rozdělení textu na termy a základní dotazy \checkmark \checkmark
		\item Elasticsearch: Výpočet skóre zásahu \checkmark \checkmark
		\item Elasticsearch: Levenštejnova vzdálenost \checkmark \checkmark
		\item Elasticsearch: Pokročilé dotazy \checkmark \checkmark
	\end{enumerate}
	
	Postup DATAB: 12/12 100\% \checkmark \checkmark
	
	\subsection*{UNIXS}
	
	\begin{enumerate}[label=\arabic*.]
		\item Unixové operační systémy (UNIX, Linux), uživatelská prostředí a nápovědy \checkmark \checkmark
		\item Unixové systémy souborů a procesů, základní programy \checkmark \checkmark
		\item Příkazový interpret (shell), vstup a výstup programu a roura v unixových systémech \checkmark \checkmark
		\item Text a regulární výrazy, základní utility, grep \checkmark \checkmark
		\item Zpracování textu v unixových OS: sed, bash \checkmark \checkmark
		\item Zpracování textu v unixových OS: awk \checkmark \checkmark
	\end{enumerate}
	
	Postup UNIXS: 6/6 100\% \checkmark \checkmark
	
	\subsection*{STRUP}
	
	\begin{enumerate}[label=\arabic*.]
		\item Architektury a princip činnosti počítače \checkmark \checkmark
		\item Čiselné soustavy \checkmark \checkmark
		\item Binární logika, logické operace a jejich vlastnosti  \checkmark \checkmark
		\item Logické funkce a jejich úpravy \checkmark \checkmark
		\item Logické obvody - kombinační \checkmark \checkmark
		\item Logické obvody - sekvenční \checkmark \checkmark
		\item Reprezentace čísel v počítači \checkmark \checkmark
		\item Reprezentace znaků v počítači \checkmark \checkmark
		\item Detekční a samoopravné kody \checkmark\checkmark
		\item Hardware osobního počítače: základní deska a chipset \checkmark \checkmark
		\item Hardware osobního počítače: procesor a instrukce \checkmark \checkmark
		\item Hardware osobního počítače: vnitřní paměti \checkmark \checkmark
		\item Hardware osobního počítače: vnější paměti \checkmark \checkmark
		\item Hardware osobního počítače: přídavné karty \checkmark \checkmark
		\item Hardware osobního počítače: ostatní zařízení \checkmark \checkmark
	\end{enumerate} 
	
	Postup STRUP: 15/15 100\% \checkmark \checkmark
	
	\subsection*{POS1}
	
	\begin{enumerate}[label=\arabic*.]
		\item Počítačové sítě, jejich služby a architektury \checkmark \checkmark
		\item Fyzická vrstva \checkmark \checkmark
		\item Ethernet: přepínač, použití média, linkový rámec \checkmark \checkmark
		\item Ethernet: STP \checkmark \checkmark
		\item Protokol IP: paket, adresy a podsítě \checkmark \checkmark
		\item Protokol IP: směrování \checkmark \checkmark
		\item Protokoly TCP a UDP: navazování a ukončení spojení \checkmark \checkmark
		\item Protokoly TCP a UDP: řešení chyb \checkmark \checkmark
		\item Protokoly TCP a UDP: řízení toku \checkmark \checkmark
		\item Systém DNS \checkmark \checkmark
		\item Aplikační služby a tvorba síťových aplikací \checkmark \checkmark
		\item Bezpečnost počítačových sítí \checkmark \checkmark
	\end{enumerate}
	
	Postup POS1: 12/12 100\% \checkmark \checkmark
	
	\subsection*{POS2}
	\begin{enumerate}[label=\arabic*.]
		\item Bezdrátové sítě: režimy, přenosové médium, problémy, bezpečnost, Bluetooth \checkmark \checkmark
		\item Wi-Fi: standardy, access point, použití média, linkový rámec, zabezpečení \checkmark \checkmark
		\item NAT: účel, typy, problémy \checkmark \checkmark
		\item IPv6: vlastnosti, paket, adresy \checkmark \checkmark
	\end{enumerate}
	
	Postup POS2: 4/4 100\% \checkmark \checkmark
	
	\subsection*{WEB}
	
	\begin{enumerate}[label=\arabic*.]
		\item Architektura webové stránky \checkmark \checkmark
		\item Syntaxe a sémantika HTML \checkmark \checkmark
		\item Syntaxe a sémantika CSS \checkmark \checkmark
		\item HTML struktura webové stránky \checkmark \checkmark
		\item Box model \checkmark \checkmark
		\item Dědičnost a kaskáda \checkmark \checkmark
		\item Základy správného psaní CSS kódů: typické chyby a metodiky \checkmark \checkmark
		\item Layout webové stránky: grid \checkmark \checkmark
		\item Layout webové stránky: flexbox \checkmark \checkmark
		\item Layout webové stránky: pozicování \checkmark \checkmark
		\item Responzivní design \checkmark \checkmark
		\item Základní HTML elementy a jejich vizualizace - text \checkmark \checkmark
		\item Základní HTML elementy a jejich vizualizace - pozadí \checkmark \checkmark
		\item Základní HTML elementy a jejich vizualizace - seznamy, tabulky, formuláře \checkmark \checkmark
		\item Analýza kvality webové stránky \checkmark \checkmark
		\item Klientský JavaScript \checkmark \checkmark
	\end{enumerate}
	
	Postup WEB: 16/16 100\% \checkmark \checkmark
	
	\subsection*{INFOS}
	
	\begin{enumerate}[label=\arabic*.]
		\item Systém: struktura, okolí, hranice, vstup a výstup, vlastnosti a klasifikace systémů \checkmark \checkmark
		\item Základní pojmy informačních systému: data, informace, informační systém \checkmark \checkmark
		\item Základní pojmy informačních systému: příklady \checkmark \checkmark
		\item Architektury informačních systémů: globální \checkmark \checkmark
		\item Architektury informačních systémů: vrstvená \checkmark \checkmark
		\item Architektury informačních systémů: integrace \checkmark \checkmark
		\item Tvorba informačních systémů: softwarový proces, metodika vývoje, analýza systému \checkmark \checkmark
		\item ER-model \checkmark \checkmark
		\item Podnikové informační systémy: popis, funkcionalita PIS-ERB, ERP II \checkmark \checkmark
		\item Business Inteligence: datový pohled na PIS vs analytické reporty \checkmark \checkmark
		\item Business Inteligence: datový sklad, OLAP \checkmark \checkmark
		\item Bezpečnost - Ochrana informací \checkmark \checkmark
		\item Bezpečnost - Zabezpečení dat \checkmark \checkmark
		\item Bezpečnost - Zavazadlová šifra, RSA, použití \checkmark \checkmark
		\item Testování (může být, podle Janoštíka častá záchranná otázka) \checkmark \checkmark
	\end{enumerate}
	
	Postup INFOS: 15/15 100\% \checkmark \checkmark
	\newline
	\newline
	Postup Informační technologie: 104/104 100\% \checkmark \checkmark
	\newline
	\newline
	1.pokus Informační technologie \checkmark
	
	\section*{Programovací jazyky a programování}
	
	\subsection*{Python:}
	
	\begin{enumerate}[label=\arabic*.]
		\item Řízení vykonávání programu v jazyce Python: bloky, cykly, větvení, funkce \checkmark \checkmark
		\item Výrazy a jejich vyhodnocování v jazyce Python \checkmark  \checkmark
		\item Základní datové typy v jazyce Python \checkmark  \checkmark
		\item Základy systému vyjímek v jazyce Python \checkmark \checkmark
		\item Typy chyb a jejich hledání v jazyce Python \checkmark \checkmark
		\item Práce se soubory v jazyce Python \checkmark \checkmark
		\item Binární data v jazyce Python \checkmark \checkmark
		\item Moduly v jazyce Python a jejich importování \checkmark \checkmark
	\end{enumerate}
	
	Postup Python: 8/8 100\% \checkmark \checkmark
	
	\subsection*{ZPP:}
	
	\begin{enumerate}[label=\arabic*.]
		\item Základy objektového programování: třídy, objekty, zasílání zpráv \checkmark \checkmark
		\item Principy objektového programování: zapouzdření, polymorfismus a dědičnost \checkmark \checkmark
		\item Události v objektovém programování \checkmark \checkmark  
		\item Funkce vyšších řádů: mapování, filtrování, redukce a anonymní funkce \checkmark \checkmark
		\item Rekurze a rekurzivní datové struktury (spojové seznamy, stromy) \checkmark \checkmark  
		\item Iterátory a generátory \checkmark \checkmark
		\item Synchronizace vláken: problém kritické sekce, zámky, semafory \checkmark \checkmark
		\item Producenti a konzumenti, večeřící filozofové \checkmark \checkmark
	\end{enumerate}
	
	Postup ZPP: 8/8 100\% \checkmark \checkmark
	
	\subsection*{Jazyk C:}
	
	\begin{enumerate}[label=\arabic*.]
		\item Přehled typového systému C \checkmark \checkmark
		\item Principy správy paměti v C \checkmark \checkmark
		\item Principy adresování a práce s pointery v C \checkmark \checkmark 
		\item Typy chyb a jejich hledání v jazyce C \checkmark \checkmark
		\item Organizace kódu v jazyce C \checkmark \checkmark
		\item Zařazení jazyka C mezi ostatní jazyky, výhody a nevýhody \checkmark \checkmark
	\end{enumerate}
	
	Postup Jazyk C: 6/6 100\% \checkmark \checkmark
	
	\subsection*{WEBA:}
	
	\begin{enumerate}[label=\arabic*.]
		\item Webové aplikace a přehled technologií používaných při jejich tvorbě \checkmark \checkmark
		\item Architektura webové aplikace a problematika škálovatelnosti \checkmark \checkmark
		\item Zpracování HTTP požadavků: předávání dat mezi webovým a aplikačním serverem, příklady realizace \checkmark \checkmark
		\item REST API: popis a příklady realizace \checkmark \checkmark
		\item JavaScript na webovém frontendu a jeho možnosti \checkmark \checkmark
		\item Technologie AJAX a její použití \checkmark \checkmark
		\item Knihovna React: charakteristika, použití \checkmark \checkmark
		\item Knihovna React: react hooks, moduly \checkmark \checkmark
		\item Možnosti tvorby nativních aplikací pomocí webových technologií \checkmark \checkmark
		\item Node.js: charakteristika použití \checkmark \checkmark
	\end{enumerate}
	
	Postup WEBA: 10/10 100\% \checkmark \checkmark
	\newline
	\newline
	Postup Programovací jazyky a programování: 32/32 100\% \checkmark \checkmark
	\newline
	\newline
	1.pokus Programovací jazyky a programování \checkmark
	\newline
	\newline
	Postup Státnicové okruhy: 188/188 100\% \checkmark \checkmark
	
	1.pokus obhajoba bakalářky
	\newline
	\newline
	Postup Státnice: 2/4 50\%
	
\end{document}
